\documentclass{article}
\usepackage[utf8]{inputenc}
\usepackage[portuguese]{babel}
\usepackage[top=1in, bottom=1in, left=1in, right=1in]{geometry}
\usepackage{graphicx}
\usepackage{indentfirst}
\usepackage{verbatim}

\setlength{\parindent}{20pt}

\title{Inteligência Artificial - Trabalho 2\\Magic Jinn}
\author{Caio Pereira Oliveira, Salomão Rodrigues Jacinto e William Kraemer Aliaga}
\date{Maio de 2017}


\begin{document}

\maketitle

\section{Descrição}

Magic Jinn é um brinquedo infantil na forma de um animal no qual ele diz ler sua mente. O objetivo é que a pessoa pense em um animal e ele fará uma sequência de perguntas que podem ser respondidas com  \textit{Sim}, \textit{Não}, \textit{Não Sei} ou \textit{Talvez} e a partir das respostas ele deduz qual era o animal que a pessoa estava pensando.

Uma ontologia é um modelo de dados que representa um conjunto de conceitos dentro de um domínio e os relacionamentos entre estes e é utilizada para realizar inferência sobre os objetos do domínio. Assim como o Magic Jinn o domínio utilizado neste trabalho são animais.


\section{Detalhes de Implementação}

As ontologias geralmente descrevem índividuos com seus relacionamentos e atributos, para o trabalho foram definidas uma série de características comuns entre os animais para conseguir extrair através das perguntas grupos de animais que podem ser o que o jogador está pensando. Dentre estas características estão:

\begin{itemize}
    \item Tipo (Peixe, Anfpibio, Réptil, Pássaro ou  Mamífero)
    \item Cor (Preto, Azul, Marrom, Verde, Cinza, Laranja, Rosa, Vermelho, Roxo, Branco, Amarelo)
    \item Pele (Couro, Pena, Pêlo, Escama, Casco, Úmida)
    \item Hábitat (Deserto, Água Doce, Planície, Montanha, Oceano, Pólo Norte/Sul, Pântano, Floresta, Urbano)
    \item Tamanho (Muito Pequeno, Pequeno, Médio, Grande)
    \item Dieta (Carnívoro, Herbívoro, Onívoro)
\end{itemize}

O grupo dos invertebrados não foi utilizado pela dificuldade em definir algumas propriedades e também foi adicionada uma característica na forma de [Sim/Não] para definir se o animal é perigoso para humanos.

%INCLUIR O CSV QUE APARECE NO GIT PRA MOSTRAR OS ANIMAIS QUE ESTAO NA BASE
%\includegraphics[width=\textwidth]{animals_csv}

\section{Algoritmos}

%talvez explicar como o pandas funciona e o que acontece quando o sistema recebe o resultado de cada pergunta

\section{Testes}

%descrever passo a passo a partir da escolha de um animal e ver o andamento do programa

\section{Execução do Programa}

%como faz pra rodar, talvez seja aqui que fale da necessidade de ter o pandas instalado e formas como pode fazer isso

\end{document}

