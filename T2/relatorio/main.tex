\documentclass{article}
\usepackage[utf8]{inputenc}
\usepackage[portuguese]{babel}
\usepackage[top=1in, bottom=1in, left=1in, right=1in]{geometry}
\usepackage{graphicx}
\usepackage{indentfirst}
\usepackage{verbatim}
\usepackage{csvsimple}


\setlength{\parindent}{20pt}

\title{Inteligência Artificial - Trabalho 2\\Jogo de Adivinhações}
\author{Caio Pereira Oliveira, Salomão Rodrigues Jacinto e William Kraemer Aliaga}
\date{Maio de 2017}


\begin{document}

\maketitle

\section{Descrição}

Magic Jinn é um brinquedo infantil na forma de um animal no qual ele diz ler sua mente. O objetivo é que a pessoa pense em um animal e ele fará uma sequência de perguntas que podem ser respondidas com  \textit{Sim}, \textit{Não}, \textit{Não Sei} ou \textit{Talvez} e a partir das respostas ele deduz qual era o animal que a pessoa estava pensando.

Uma ontologia é um modelo de dados que representa um conjunto de conceitos dentro de um domínio e os relacionamentos entre estes e é utilizada para realizar inferência sobre os objetos do domínio. Assim como o Magic Jinn o domínio utilizado neste trabalho são animais.


\section{Detalhes de Implementação}

As ontologias geralmente descrevem indivíduos com seus relacionamentos e atributos, para o trabalho foram definidas uma série de características comuns entre os animais para conseguir extrair através das perguntas grupos de animais que podem ser o que o jogador está pensando. Dentre estas características estão:

\begin{itemize}
    \item Tipo (Peixe, Anfíbio, Réptil, Pássaro ou  Mamífero)
    \item Cor (Preto, Azul, Marrom, Verde, Cinza, Laranja, Rosa, Vermelho, Roxo, Branco, Amarelo)
    \item Pele (Couro, Pena, Pelo, Escama, Casco, Úmida)
    \item Hábitat (Deserto, Água Doce, Planície, Montanha, Oceano, Polo Norte/Sul, Pântano, Floresta, Urbano)
    \item Tamanho (Muito Pequeno, Pequeno, Médio, Grande)
    \item Dieta (Carnívoro, Herbívoro, Onívoro)
\end{itemize}

O grupo dos invertebrados não foi utilizado pela dificuldade em definir algumas propriedades e também foi adicionada uma característica na forma de [Sim/Não] para definir se o animal é perigoso para humanos. Abaixo segue a tabela com todos os animais cadastrados:

%TABELAS
\pagebreak
\newgeometry{top=1in, bottom=1in, left=0.3in, right=1in}
\csvautotabular[respect all]{animals.csv}

\csvautotabular[respect all]{animals2.csv}
\newgeometry{top=1in, bottom=1in, left=1in, right=1in}
\pagebreak


\section{Algoritmos}

Para este trabalho, foi utilizada a biblioteca \textit{pandas}, muito utilizada em ciência de dados, que permite fácil manipulação de dados tabulares.

O principal algoritmo utilizado funciona da seguinte maneira:

\begin{verbatim}
inicie uma lista de grupos vazia

para cada propriedade p
    agrupe a tabela pela propriedade p (group by)
    
    encontre o maior grupo e adicione-o a lista de grupos

fim para

encontre o maior grupo na lista de grupos e retorne-o
\end{verbatim}

Este algoritmo é usado encontrar o maior grupo de animais com uma certa característica, então o usuário é questionado se o animal que escolheu possui tal característica.

Se responder positivamente, todos os animais que não possuem esta característica são descartados, senão, todos os animais que possuírem essa característica são descartados.


\section{Exemplo de Execução}

Abaixo segue como seguiria o jogo se o animal imaginado fosse uma Iguana:

\begin{verbatim}
Responda com "y" para Sim e "n" para Não
Qualquer outra resposta será interpretada como "não sei/talvez"

O animal é perigoso para humanos?
n
O animal é um mamífero?
n
O animal tem entre 10 centímetros e 1 metro?
y
O animal é onívoro?
n
O habitat do animal é "Floresta"?
y
Todos os animais desta espécie possuem a cor verde?
y
O animal é um anfíbio?
n
O animal é herbívoro?
y
O animal é um ave?
n
O animal é Iguana!
Fim
\end{verbatim}


\pagebreak


\section{Execução do Programa}

Como já mencionado anteriormente o programa utiliza da biblioteca \textit{pandas} que pode ser instalada utilizando \textit{pip} na maioria dos computadores com ambiente linux:

\begin{verbatim}
sudo pip3 install pandas==0.20.1
\end{verbatim}

Para rodar o programa você precisa dos arquivos disponibilizados \textit{animals.py}, \textit{classifier.py}, \textit{questions.py} e \textit{main.py} e digitar o comando:

\begin{verbatim}
python3 main.py
\end{verbatim}

\end{document}
